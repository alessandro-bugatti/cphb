\chapter{Strutture dati}

\index{struttura dati}

Una \key{struttura dati} è un modo per 
memorizzare dei dati nella memoria di un computer.
È importante scegliere la struttura dati appropriata
per il problema che si intende risolvere,
poichè ogni struttura ha i suoi vantaggi
e svantaggi.
La domanda fondamentale da porsi è: quali operazioni
sono efficienti data una certa struttura?

In questo capitolo verranno mostrate le più
importanti strutture dati della libreria standard
del C++.
È una buona idea usare la libreria standard 
tutte le volte che ciò sia possibile,
perchè in questo modo è possibile risparmiare
una sacco di tempo.
Nel seguito verranno anche mostrate strutture più
sofisticate che non sono disponibili nella libreria standard.


\section{Array dinamici}

\index{array dinamico}
\index{vettore}

Un \key{array dinamico} è un array la cui dimensione
può cambiare durante l'esecuzione del programma.
L'array dinamico più conosciuto in C++
è la struttura \texttt{vector},
che può essere utilizzato quasi come fosse un normale array.

Il seguente codice crea un vettore vuoto e
aggiunge tre elementi:

\begin{lstlisting}
vector<int> v;
v.push_back(3); // [3]
v.push_back(2); // [3,2]
v.push_back(5); // [3,2,5]
\end{lstlisting}

Fatto ciò, è possibile accedere ai tre elementi come se si trattasse di un normale array: 

\begin{lstlisting}
cout << v[0] << "\n"; // 3
cout << v[1] << "\n"; // 2
cout << v[2] << "\n"; // 5
\end{lstlisting}

La funzione \texttt{size} ritorna il numero di elementi nel vettore.
Il seguente codice itera attraverso tutti gli elementi
del vettore e li stampa uno dopo l'altro:

\begin{lstlisting}
for (int i = 0; i < v.size(); i++) {
    cout << v[i] << "\n";
}
\end{lstlisting}

\begin{samepage}
Un modo più compatto di iterare attraverso un vettore è il seguente:

\begin{lstlisting}
for (auto x : v) {
    cout << x << "\n";
}
\end{lstlisting}
\end{samepage}

La funzione \texttt{back} ritorna l'ultimo elemento del vettore, e la 
funzione \texttt{pop\_back} rimuove l'ultimo elemento:

\begin{lstlisting}
vector<int> v;
v.push_back(5);
v.push_back(2);
cout << v.back() << "\n"; // 2
v.pop_back();
cout << v.back() << "\n"; // 5
\end{lstlisting}

Il seguente codice crea un vettore di 5 elementi:

\begin{lstlisting}
vector<int> v = {2,4,2,5,1};
\end{lstlisting}

Un altro modo di creare un vettore è di passare il numero 
di elementi e, opzionalmente, il valore iniziale di ogni elemento:

\begin{lstlisting}
// dimensione 10, valore iniziale 0
vector<int> v(10);
\end{lstlisting}
\begin{lstlisting}
// dimensione 10, valore iniziale 5
vector<int> v(10, 5);
\end{lstlisting}

L'implementazione interna del vettore 
usa un normale array. 
Se la dimensione del vettore aumenta
e l'array non riesce più a contenere tutti gli elementi,
viene allocato un nuovo array e tutti gli elementi
del vecchio array sono copiati nel nuovo array.
Comunque questa operazione non avviene spesso e la 
complessità media dell'inserimento con la \texttt{push\_back} è di tipo
$O(1)$.

\index{stringa}

La struttura \texttt{stringa} è anch'essa un array dinamico
e può essere usata quasi come fosse un vector.
Inoltre le stringhe sono dotate di una speciale sintassi
che non si ritrova nelle altre strutture dati.
Le stringhe possono essere concatenate usando il simbolo \texttt{+}.
La funzione $\texttt{substr}(k,x)$ ritorna la sottostringa
che inizia alla posizione $k$ e ha lunghezza $x$,
e la funzione $\texttt{find}(\texttt{t})$ trova la posizione della prima
occorrenza della stringa \texttt{t} all'interno della stringa.

Il seguente codice illustra alcune operazioni sulle stringhe:

\begin{lstlisting}
string a = "hatti";
string b = a+a;
cout << b << "\n"; // hattihatti
b[5] = 'v';
cout << b << "\n"; // hattivatti
string c = b.substr(3,4);
cout << c << "\n"; // tiva
\end{lstlisting}

\section{Set}

\index{set}

Un \key{set} è una struttura dati che contiene un 
insieme o collezione di elementi.
Le operazioni base che si possono applicare a un set
sono l'inserimento, la ricerca e la rimozione. 
La libreria standard del C++ contiene due implementazioni
dei set:
\begin{itemize}
\item la struttura \texttt{set}, che è basata su un albero binario bilanciato e 
le cui operazioni hanno un costo di tipo $O(\log n)$.
\item la struttura \texttt{unordered\_set} che invece usa l'hashing,
e le sue operazioni hanno in media prestazioni di tipo $O(1)$.
\end{itemize}

La scelta su quale implementazione utilizzare
è spesso solo una questione di "gusti".
Il beneficio della struttura \texttt{set} 
è che mantiene l'ordine degli elementi inseriti e
che espone funzioni che non sono disponibili negli
\texttt{unordered\_set}.
D'altronde gli \texttt{unordered\_set}
possono essere più efficienti.

Il seguente codice crea un set che contiene interi e mostra
alcune operazioni basilari.
La funzione \texttt{insert} aggiunge un elemento al set,
la funzione \texttt{count} ritorna il numero di occorrenze 
di un elemento all'interno di un set
e la funzione \texttt{erase} rimuove un elemento dal set.

\begin{lstlisting}
set<int> s;
s.insert(3);
s.insert(2);
s.insert(5);
cout << s.count(3) << "\n"; // 1
cout << s.count(4) << "\n"; // 0
s.erase(3);
s.insert(4);
cout << s.count(3) << "\n"; // 0
cout << s.count(4) << "\n"; // 1
\end{lstlisting}

Un set può essere utilizzato più o meno come un vettore,
ma non è possibile accedere agli elementi
usando la notazione \texttt{[]}.
Il seguente codice crea un set,
stampa il numero di elementi contenuti e itera
attraverso tutti gli elementi:
\begin{lstlisting}
set<int> s = {2,5,6,8};
cout << s.size() << "\n"; // 4
for (auto x : s) {
    cout << x << "\n";
}
\end{lstlisting}

Un'importante proprietà dei set è che tutti gli elementi
sono \emph{distinti}, quindi la funzione \texttt{count}
ritorna o 0 (l'elemento non è presente) oppure 1 
(l'elemento è presente) e la funzione 
\texttt{insert} non aggiunge mai un elemento
se è già presente nel set, come si può vedere
nel seguente codice:

\begin{lstlisting}
set<int> s;
s.insert(5);
s.insert(5);
s.insert(5);
cout << s.count(5) << "\n"; // 1
\end{lstlisting}

Inoltre il C++ contiene le strutture
\texttt{multiset} e \texttt{unordered\_multiset}
che funzionano come i \texttt{set}
e gli \texttt{unordered\_set}
tranne che possono contenere più istanze dello stesso elemento.
Per esempio nel seguente codice tutte e tre le istanze del
numero 5 vengono aggiunte al multiset:

\begin{lstlisting}
multiset<int> s;
s.insert(5);
s.insert(5);
s.insert(5);
cout << s.count(5) << "\n"; // 3
\end{lstlisting}
La funzione \texttt{erase} rimuove tutte
le istanze di un elemento da un multiset:
\begin{lstlisting}
s.erase(5);
cout << s.count(5) << "\n"; // 0
\end{lstlisting}
Quando invece deve essere rimossa una sola istanza
di un elemento da un multiset,
è possibile procedere in questo modo:
\begin{lstlisting}
s.erase(s.find(5));
cout << s.count(5) << "\n"; // 2
\end{lstlisting}

\section{Mappe}

\index{map}

Una \key{map} è un array generalizzato
che consiste di un insieme di coppie chiave-valore.
Mentre in un array normale le chiavi sono sempre
degli interi consecutivi $0,1,\ldots,n-1$,
dove $n$ è la dimensione del vettore,
le chiavi in una mappa possono essere di qualsiasi tipo
e non devono avere valori consecutivi.

La libreria standard del C++ contiene
due implementazioni di mappe, che corrispondono
alle rispettive implementazioni dei set: la struttura
\texttt{map} è basata su un albero binario bilanciato
e accedere a un elemento ha costo $O(\log n)$,
mentre la struttura
\texttt{unordered\_map} utilizza l'hashing
e l'accesso a ogni elemento costa in media $O(1)$.

Il seguente codice crea una mappa
dove le chiavi sono stringhe e i valori sono interi:

\begin{lstlisting}
map<string,int> m;
m["monkey"] = 4;
m["banana"] = 3;
m["harpsichord"] = 9;
cout << m["banana"] << "\n"; // 3
\end{lstlisting}

Se viene richiesta una chiave nella mappa
e questa non è contenuta,
viene aggiunta automaticamente con un valore 
di default che dipende dal tipo di dato.
Per esempio, nel seguente codice,
viene aggiunta alla mappa la chiave
''aybabtu'' con valore 0.

\begin{lstlisting}
map<string,int> m;
cout << m["aybabtu"] << "\n"; // 0
\end{lstlisting}

La funzione \texttt{count} controlla
se esiste una chiave nella mappa:

\begin{lstlisting}
if (m.count("aybabtu")) {
    // la chiave esiste
}
\end{lstlisting}
Il codice seguente stampa tutte le chiavi con i rispettivi valori associati:
\begin{lstlisting}
for (auto x : m) {
    cout << x.first << " " << x.second << "\n";
}
\end{lstlisting}

\section{Iteratori e range}

\index{iterator}

Molte funzioni della libreria standard del C++
usano gli iteratori.
Un \key{iteratore} è una variabile
che punta a un elemento in una struttura dati.

Spesso vengono utilizzati gli iteratori \texttt{begin}
e \texttt{end} che definiscono un range (intervallo)
che contiene tutti gli elementi di una struttura dati.
L'iteratore \texttt{begin} punta al primo elemento 
della struttura dati,
mentre l'iteratore \texttt{end} punta alla posizione \emph{dopo}
l'ultimo elemento.
Questa è la situazione che si presenta:

\begin{center}
\begin{tabular}{llllllllll}
\{ & 3, & 4, & 6, & 8, & 12, & 13, & 14, & 17 & \} \\
& $\uparrow$ & & & & & & & & $\uparrow$ \\
& \multicolumn{3}{l}{\texttt{s.begin()}} & & & & & & \texttt{s.end()} \\
\end{tabular}
\end{center}

Come si può vedere gli iteratori sono asimmetrici:
\texttt{s.begin()} punta a un elemento contenuto nella struttura dati,
mentre \texttt{s.end()} punta al di fuori della struttura dati.
Quindi l'intervallo definito dagli operatori è \emph{aperto a destra}.

\subsubsection{Lavorare con gli intervalli}

Gli iteratori sono spesso usati nelle funzioni della libreria
standard che necessitano di un intervallo all'interno di una struttura dati.
Una delle cose che si vogliono fare normalmente è elaborare
tutti gli elementi all'interno di una struttura dati,
così alla funzione verranno passati gli iteratori
\texttt{begin} e \texttt{end}.

Per esempio, il seguente codice ordina un vettore
usando la funzione \texttt{sort},
poi inverte l'ordine degli elementi usando la funzione
\texttt{reverse} e infine mescola gli elementi usando la funzione
\texttt{random\_shuffle}.

\index{sort@\texttt{sort}}
\index{reverse@\texttt{reverse}}
\index{random\_shuffle@\texttt{random\_shuffle}}

\begin{lstlisting}
sort(v.begin(), v.end());
reverse(v.begin(), v.end());
random_shuffle(v.begin(), v.end());
\end{lstlisting}

Queste funzioni possono anche essere utilizzate con normali array.
In questo caso verranno usati dei puntatori al posto degli iteratori:

\newpage
\begin{lstlisting}
//sia a ad esempio un vettore di interi
//int a[n];
sort(a, a+n);
reverse(a, a+n);
random_shuffle(a, a+n);
\end{lstlisting}

\subsubsection{Iteratori su insiemi}

Gli iteratori sono spesso usati per accedere 
agli elementi di un set.
Il seguente codice crea un iteratore
\texttt{it} che punta al più piccolo elemento di un set:
\begin{lstlisting}
set<int>::iterator it = s.begin();
\end{lstlisting}
Un modo più compatto di scrivere la stessa cosa è il seguente:
\begin{lstlisting}
auto it = s.begin();
\end{lstlisting}
L'elemento a cui punta un iteratore può essere acceduto
tramite l'operatore \texttt{*}.
Per esempio il seguente codice
stampa il più piccolo elemento nel set:

\begin{lstlisting}
auto it = s.begin();
cout << *it << "\n";
\end{lstlisting}

Gli iteratori possono essere mossi usando gli operatori
\texttt{++} (in avanti) e \texttt{--} (all'indietro),
cioè l'iteratore si sposterà all'elemento successivo (++)
o a quello precedente (- -).

Il seguente codice stampa tutti gli elementi in ordine crescente:
\begin{lstlisting}
for (auto it = s.begin(); it != s.end(); it++) {
    cout << *it << "\n";
}
\end{lstlisting}
Il seguente codice stampa il più grande elemento presente nell'insieme:
\begin{lstlisting}
auto it = s.end(); it--;
cout << *it << "\n";
\end{lstlisting}

La funzione $\texttt{find}(x)$ ritorna un iteratore che punta
all'elemento il cui valore è $x$.
Se però l'insieme non contiene $x$,
l'iteratore diventerà \texttt{end}.

\begin{lstlisting}
auto it = s.find(x);
if (it == s.end()) {
    // x non presente
}
\end{lstlisting}

La funzione $\texttt{lower\_bound}(x)$ ritorna 
un iteratore al più piccolo elemento nell'insieme
il cui valore è \emph{almeno} $x$, e 
la funzione $\texttt{upper\_bound}(x)$
ritorna un iteratore al più piccolo elemento il cui
valore è \emph{maggiore} di $x$.
Entrambe le funzioni, se tale elemento non esiste,
ritornano l'iteratore \texttt{end}.
Queste funzioni non sono implementate per 
l'\texttt{unordered\_set}, poichè questo non 
mantiene l'ordine degli elementi.

\begin{samepage}
Per esempio, il seguente codice trova l'elemento più vicino a $x$:

\begin{lstlisting}
auto it = s.lower_bound(x);
if (it == s.begin()) {
    cout << *it << "\n";
} else if (it == s.end()) {
    it--;
    cout << *it << "\n";
} else {
    int a = *it; it--;
    int b = *it;
    if (x-b < a-x) cout << b << "\n";
    else cout << a << "\n";
}
\end{lstlisting}

Questo codice assume che il set non sia vuoto,
e analizza tutti i casi possibili guardando il valore 
dell'iteratore \texttt{it}.
Siccome \texttt{it} punta al più piccolo 
valore che è almeno $x$, ci sono tre possibilità:
\begin{itemize}
\item se \texttt{it} è uguale a \texttt{begin},
l'elemento corrispondente è il più vicino a $x$ 
\item se \texttt{it} è uguale a \texttt{end},
l'elemento più grande del set è il più vicino a $x$ 
\item se non si verifica nessuno dei due casi precedenti,
l'elemento più vicino a $x$ è o l'elemento puntato da
\texttt{it} oppure l'elemento precedente.
\end{itemize}
\end{samepage}

\section{Altre strutture}

\subsubsection{Bitset}

\index{bitset}

Un \key{bitset} è un array in cui ogni valore
è 0 o 1. 
Per esempio il seguente codice crea
un bitset che contiene 10 elementi:
\begin{lstlisting}
bitset<10> s;
s[1] = 1;
s[3] = 1;
s[4] = 1;
s[7] = 1;
cout << s[4] << "\n"; // 1
cout << s[5] << "\n"; // 0
\end{lstlisting}

Il vantaggio di usare i bitset è che 
richiedono meno memoria che gli array normali,
perchè ogni elemento in un bitset usa
un solo bit di memoria.
Per esempio, se $n$ bit sono memorizzati in un
array di interi, saranno usati $32n$ bit di memoria,
mentre il bitset contenente le stesse informazioni
richiederà solamente $n$ bit di memoria.
Inoltre i valori all'interno di un bitset 
possono essere manipolati in maniera efficiente
usando gli operatori sui bit,
rendendo possibile ottimizzare gli algoritmi
che utilizzano dei bitset.

Il seguente codice mostra un altro modo di creare il bitset
visto in precedenza:
\begin{lstlisting}
bitset<10> s(string("0010011010")); // from right to left
cout << s[4] << "\n"; // 1
cout << s[5] << "\n"; // 0
\end{lstlisting}

La funzione \texttt{count} ritorna il numero 
di 1 contenuti nel bitset:

\begin{lstlisting}
bitset<10> s(string("0010011010"));
cout << s.count() << "\n"; // 4
\end{lstlisting}

Il seguente codice mostra degli esempi di operazioni sui bit:
\begin{lstlisting}
bitset<10> a(string("0010110110"));
bitset<10> b(string("1011011000"));
cout << (a&b) << "\n"; // 0010010000, AND
cout << (a|b) << "\n"; // 1011111110, OR
cout << (a^b) << "\n"; // 1001101110, XOR
\end{lstlisting}

\subsubsection{Deque}

\index{deque}

Una \key{deque} è un array dinamico
la cui dimensione può essere modificata in 
maniera efficiente ad entrambi gli estremi dell'array.
Una deque provvede delle funzioni uguali a quelle di un vector,
come \texttt{push\_back} e \texttt{pop\_back}, 
inoltre però possiede anche le funzioni
\texttt{push\_front} e \texttt{pop\_front}
che non sono disponibili nei vector.

Una deque può essere usata nel modo seguente:
\begin{lstlisting}
deque<int> d;
d.push_back(5); // [5]
d.push_back(2); // [5,2]
d.push_front(3); // [3,5,2]
d.pop_back(); // [3,5]
d.pop_front(); // [5]
\end{lstlisting}

L'implementazione interna di una deque è più 
complessa di quella di un vettore,
e per questa ragione è più lenta di un vettore.
Comunque, sia aggiungere che rimuovere elementi 
da entrambi gli estremi in media è un'operazione di tipo $O(1)$.

\subsubsection{Stack}

\index{stack}

Uno \key{stack} è una struttura dati che fornisce 
due operazione che hanno costo $O(1)$:
aggiungere un elemento un elemento in testa
e rimuovere un elemento in testa.
Quindi in uno stack l'accesso avviene da una sola
estremità e solo l'elemento in quella posizione può essere 
acceduto tramite il metodo \key{top}.

Il seguente codice mostra come può essere utilizzato uno stack:
\begin{lstlisting}
stack<int> s;
s.push(3);
s.push(2);
s.push(5);
cout << s.top(); // 5
s.pop();
cout << s.top(); // 2
\end{lstlisting}
\subsubsection{Queue}

\index{queue}

Anche una \key{queue} (coda in italiano) 
espone due operazioni di tipo $O(1)$:
aggiungere un elemento alla fine della coda,
e rimuovere il primo elemento della coda.
Quindi gli unici elementi a cui si può avere 
accesso sono il primo e l'ultimo elemento 
della coda.

Il seguente codice mostra un possibile utilizzo di una coda:
\begin{lstlisting}
queue<int> q;
q.push(3);
q.push(2);
q.push(5);
cout << q.front(); // 3
q.pop();
cout << q.front(); // 2
\end{lstlisting}

\subsubsection{Priority queue (coda con priorità)}

\index{priority queue}
\index{heap}

Una \key{priority queue} (coda con priorità in italiano)
contiene un insieme di elementi.
Le operazioni supportate sono l'inserimento e,
a seconda del tipo di coda con priorità,
il recupero e la rimozione del massimo o minimo elemento 
della coda.
L'inserimento e la rimozione sono operazioni
con costo $O(\log n)$,
invece il recupero dell'elemento massimo o minimo
è di tipo $O(1)$.

Mentre un set ordinato supporta in maniera
efficiente tutte le operazioni che supporta una coda con priorità,
il beneficio nell'utilizzare una coda con priorità è che
ha un fattore costante più piccolo, quindi è leggermente più efficiente 
nell'operazione di recupero. Questo perchè una coda con priorità 
è generalmente implementata utilizzando un heap, che è una 
struttura molto più semplice che l'albero binario bilanciato
solitamente usato nei set ordinati.

\begin{samepage}
Di default gli elementi di una coda con priorità
del C++ sono ordinati in ordine decrescente e quindi è
possibile recuperare e eliminare il massimo
elemento presente.
Il seguente codice mostra come farlo:

\begin{lstlisting}
priority_queue<int> q;
q.push(3);
q.push(5);
q.push(7);
q.push(2);
cout << q.top() << "\n"; // 7
q.pop();
cout << q.top() << "\n"; // 5
q.pop();
q.push(6);
cout << q.top() << "\n"; // 6
q.pop();
\end{lstlisting}
\end{samepage}

Volendo creare una coda con priorità 
che supporti il recupero e la rimozione dell'elemento
più piccolo si può procedere in questo modo:

\begin{lstlisting}
priority_queue<int,vector<int>,greater<int>> q;
\end{lstlisting}

\subsubsection{Strutture dati 'policy-based'}

Il compilatore \texttt{g++} supporta anche 
alcune strutture dati che non sono parte della libreria
standard del C++.
Queste strutture sono chiamate \emph{policy-based}.
Per usare queste strutture è necessario aggiungere 
le seguenti linee di codice:
\begin{lstlisting}
#include <ext/pb_ds/assoc_container.hpp>
using namespace __gnu_pbds; 
\end{lstlisting}
Dopo aver inserito queste righe è possibile definire una struttura
dati \texttt{indexed\_set} che è come un 
\texttt{set} ma può essere indicizzata come un array.
La definizione per i valori \texttt{int} è la seguente:

\begin{lstlisting}
typedef tree<int,null_type,less<int>,rb_tree_tag,
             tree_order_statistics_node_update> indexed_set; 
\end{lstlisting}

A questo punto è possibile dichiarare un indexed\_set in questo modo:
\begin{lstlisting}
indexed_set s;
s.insert(2);
s.insert(3);
s.insert(7);
s.insert(9);
\end{lstlisting}

La caratteristica di questo particolare set è che permette l'accesso
tramite indici come se si trattasse di un vettore ordinato.
La funzione $\texttt{find\_by\_order}$ ritorna un iteratore
all'elemento in una particolare posizione:
\begin{lstlisting}
auto x = s.find_by_order(2);
cout << *x << "\n"; // 7
\end{lstlisting}
E la funzione $\texttt{order\_of\_key}$
ritorna la posizione di un dato elemento:

\begin{lstlisting}
cout << s.order_of_key(7) << "\n"; // 2
\end{lstlisting}
Se l'elemento non è presente nel set,
viene ritornana la posizione che l'elemento avrebbe nel set:
\begin{lstlisting}
cout << s.order_of_key(6) << "\n"; // 2
cout << s.order_of_key(8) << "\n"; // 3
\end{lstlisting}
Entrambe le funzioni hanno un tempo di esecuzione di tipo logaritmico.

\section{Confronto e ordinamento}

Spesso è possibile risolvere un problema 
usando delle strutture dati oppure l'ordinamento.
A volte però possono esserci delle differenze notevoli
tra questi diversi approci, differenze che rimangono
nascoste quando si guarda solo alla complessità computazionale.

Si consideri ad esempio un problema nel
quale siano date due liste $A$ e $B$,
entrambe contenenti $n$ elementi.
Il compito è di trovare quanti sono gli elementi
contenuti in entrambe le liste.
Per esempio, date le liste
\[A = [5,2,8,9,4] \hspace{10px} \textrm{e} \hspace{10px} B = [3,2,9,5],\]
la risposta è 3 perchè 2, 5 e 9 appartengono 
a entrambe le liste.

Una soluzione diretta del problema è quella di
confrontare tutte le possibili copie di elementi 
presi da A e B, con un costo di tipo $O(n^2)$,
ma come si vedrà nel seguito ci sono algoritmi
molto più efficienti.

\subsubsection{Algoritmo 1}

Si costruisca il set degli elementi appartenenti a $A$,
e successivamente si iteri tra gli elementi di $B$ 
e si controlli per ogni elemento se appartiene anche a $A$.
Questo è efficiente perchè gli elementi di $A$
si trovano in un set.
Usando la struttura \texttt{set},
la complessità di questo algoritmo diventa di tipo $O(n \log n)$.

\subsubsection{Algoritmo 2}

Siccome non è necessario mantenere il set ordinato,
al posto della struttura \texttt{set} 
si può provare a utilizzare un \texttt{unordered\_set}.
Questo è un modo molto semplice per ottimizzare 
le prestazioni dell'algoritmo, perchè basta cambiare la
struttura dati sottostante.
La complessità del nuovo algoritmo diventa quindi $O(n)$.

\subsubsection{Algoritmo 3}
Al posto delle strutture dati è possibile utilizzare
l'ordinamento.
Come primo passo è necessario ordinare entrambe le liste $A$ e $B$
e successivamente iterare attraverso le liste allo stesso tempo
per trovare gli elementi comuni.
La complessità dell'ordinamento è $O(n \log n)$,
e il resto dell'algoritmo funziona in tempo $O(n)$,
così la complessità totale sarà $O(n \log n)$.

\subsubsection{Confronto di prestazioni}

La seguente tabella mostra quanto siano efficienti
gli algoritmi appena visti quando $n$ varia e
gli elementi della lista sono interi casuali con valori
compresi tra $1 \ldots 10^9$:

\begin{center}
\begin{tabular}{rrrr}
$n$ & Algoritmo 1 & Algoritmo 2 & Algoritmo 3 \\
\hline
$10^6$ & $1.5$ s & $0.3$ s & $0.2$ s \\
$2 \cdot 10^6$ & $3.7$ s & $0.8$ s & $0.3$ s \\
$3 \cdot 10^6$ & $5.7$ s & $1.3$ s & $0.5$ s \\
$4 \cdot 10^6$ & $7.7$ s & $1.7$ s & $0.7$ s \\
$5 \cdot 10^6$ & $10.0$ s & $2.3$ s & $0.9$ s \\
\end{tabular}
\end{center}

Gli algoritmi 1 e 2 sono uguali tranne che per il 
fatto che usano strutture dati differenti.
In questo problema, questa scelta ha un effetto
importante sul tempo di esecuzione, perchè l'algoritmo 2
è 4-5 volte più veloce dell'algoritmo 1.

Comunque l'algoritmo 3, che usa l'ordinamento, risulta essere
il più efficiente di tutti, impiegando solo metà del tempo
dell'algoritmo 2.
È interessante notare che sia l'algoritmo 1 che il 3
hanno complessità computazionale uguale a $O(n \log n)$,
ma, nonostante questo, l'algoritmo 3 è dieci volte più veloce.
Questo può essere spiegato dal fatto che l'ordinamento è una
procedura semplice e viene effettuata una sola volta all'inizio
dell'algoritmo 3, dopodichè l'algoritmo lavora in 
tempo lineare.
D'altra parte l'algoritmo 1 deve mantenere bilanciato 
un albero binario durante tutto il suo tempo d'esecuzione.

\chapter{Strutture dati}

\index{struttura dati}

Una \key{struttura dati} è un modo per 
memorizzare dei dati nella memoria di un computer.
È importante scegliere la struttura dati appropriata
per il problema che si intende risolvere,
poichè ogni struttura ha i suoi vantaggi
e svantaggi.
La domanda fondamentale da porsi è: quali operazioni
sono efficienti data una certa struttura?

In questo capitolo verranno mostrate le più
importanti strutture dati della libreria standard
del C++.
È una buona idea usare la libreria standard 
tutte le volte che ciò sia possibile,
perchè in questo modo è possibile risparmiare
una sacco di tempo.
Nel seguito verranno anche mostrate strutture più
sofisticate che non sono disponibili nella libreria standard.


\section{Array dinamici}

\index{array dinamico}
\index{vettore}

Un \key{array dinamico} è un array la cui dimensione
può cambiare durante l'esecuzione del programma.
L'array dinamico più conosciuto in C++
è la struttura \texttt{vector},
che può essere utilizzato quasi come fosse un normale array.

Il seguente codice crea un vettore vuoto e
aggiunge tre elementi:

\begin{lstlisting}
vector<int> v;
v.push_back(3); // [3]
v.push_back(2); // [3,2]
v.push_back(5); // [3,2,5]
\end{lstlisting}

Fatto ciò, è possibile accedere ai tre elementi come se si trattasse di un normale array: 

\begin{lstlisting}
cout << v[0] << "\n"; // 3
cout << v[1] << "\n"; // 2
cout << v[2] << "\n"; // 5
\end{lstlisting}

La funzione \texttt{size} ritorna il numero di elementi nel vettore.
Il seguente codice itera attraverso tutti gli elementi
del vettore e li stampa uno dopo l'altro:

\begin{lstlisting}
for (int i = 0; i < v.size(); i++) {
    cout << v[i] << "\n";
}
\end{lstlisting}

\begin{samepage}
Un modo più compatto di iterare attraverso un vettore è il seguente:

\begin{lstlisting}
for (auto x : v) {
    cout << x << "\n";
}
\end{lstlisting}
\end{samepage}

La funzione \texttt{back} ritorna l'ultimo elemento del vettore, e la 
funzione \texttt{pop\_back} rimuove l'ultimo elemento:

\begin{lstlisting}
vector<int> v;
v.push_back(5);
v.push_back(2);
cout << v.back() << "\n"; // 2
v.pop_back();
cout << v.back() << "\n"; // 5
\end{lstlisting}

Il seguente codice crea un vettore di 5 elementi:

\begin{lstlisting}
vector<int> v = {2,4,2,5,1};
\end{lstlisting}

Un altro modo di creare un vettore è di passare il numero 
di elementi e, opzionalmente, il valore iniziale di ogni elemento:

\begin{lstlisting}
// dimensione 10, valore iniziale 0
vector<int> v(10);
\end{lstlisting}
\begin{lstlisting}
// dimensione 10, valore iniziale 5
vector<int> v(10, 5);
\end{lstlisting}

L'implementazione interna del vettore 
usa un normale array. 
Se la dimensione del vettore aumenta
e l'array non riesce più a contenere tutti gli elementi,
viene allocato un nuovo array e tutti gli elementi
del vecchio array sono copiati nel nuovo array.
Comunque questa operazione non avviene spesso e la 
complessità media dell'inserimento con la \texttt{push\_back} è di tipo
$O(1)$.

\index{stringa}

La struttura \texttt{stringa} è anch'essa un array dinamico
e può essere usata quasi come fosse un vector.
Inoltre le stringhe sono dotate di una speciale sintassi
che non si ritrova nelle altre strutture dati.
Le stringhe possono essere concatenate usando il simbolo \texttt{+}.
La funzione $\texttt{substr}(k,x)$ ritorna la sottostringa
che inizia alla posizione $k$ e ha lunghezza $x$,
e la funzione $\texttt{find}(\texttt{t})$ trova la posizione della prima
occorrenza della stringa \texttt{t} all'interno della stringa.

Il seguente codice illustra alcune operazioni sulle stringhe:

\begin{lstlisting}
string a = "hatti";
string b = a+a;
cout << b << "\n"; // hattihatti
b[5] = 'v';
cout << b << "\n"; // hattivatti
string c = b.substr(3,4);
cout << c << "\n"; // tiva
\end{lstlisting}

\section{Set}

\index{set}

Un \key{set} è una struttura dati che contiene un 
insieme o collezione di elementi.
Le operazioni base che si possono applicare a un set
sono l'inserimento, la ricerca e la rimozione. 
La libreria standard del C++ contiene due implementazioni
dei set:
\begin{itemize}
\item la struttura \texttt{set}, che è basata su un albero binario bilanciato e 
le cui operazioni hanno un costo di tipo $O(\log n)$.
\item la struttura \texttt{unordered\_set} che invece usa l'hashing,
e le sue operazioni hanno in media prestazioni di tipo $O(1)$.
\end{itemize}

La scelta su quale implementazione utilizzare
è spesso solo una questione di "gusti".
Il beneficio della struttura \texttt{set} 
è che mantiene l'ordine degli elementi inseriti e
che espone funzioni che non sono disponibili negli
\texttt{unordered\_set}.
D'altronde gli \texttt{unordered\_set}
possono essere più efficienti.

Il seguente codice crea un set che contiene interi e mostra
alcune operazioni basilari.
La funzione \texttt{insert} aggiunge un elemento al set,
la funzione \texttt{count} ritorna il numero di occorrenze 
di un elemento all'interno di un set
e la funzione \texttt{erase} rimuove un elemento dal set.

\begin{lstlisting}
set<int> s;
s.insert(3);
s.insert(2);
s.insert(5);
cout << s.count(3) << "\n"; // 1
cout << s.count(4) << "\n"; // 0
s.erase(3);
s.insert(4);
cout << s.count(3) << "\n"; // 0
cout << s.count(4) << "\n"; // 1
\end{lstlisting}

Un set può essere utilizzato più o meno come un vettore,
ma non è possibile accedere agli elementi
usando la notazione \texttt{[]}.
Il seguente codice crea un set,
stampa il numero di elementi contenuti e itera
attraverso tutti gli elementi:
\begin{lstlisting}
set<int> s = {2,5,6,8};
cout << s.size() << "\n"; // 4
for (auto x : s) {
    cout << x << "\n";
}
\end{lstlisting}

Un'importante proprietà dei set è che tutti gli elementi
sono \emph{distinti}, quindi la funzione \texttt{count}
ritorna o 0 (l'elemento non è presente) oppure 1 
(l'elemento è presente) e la funzione 
\texttt{insert} non aggiunge mai un elemento
se è già presente nel set, come si può vedere
nel seguente codice:

\begin{lstlisting}
set<int> s;
s.insert(5);
s.insert(5);
s.insert(5);
cout << s.count(5) << "\n"; // 1
\end{lstlisting}

Inoltre il C++ contiene le strutture
\texttt{multiset} e \texttt{unordered\_multiset}
che funzionano come i \texttt{set}
e gli \texttt{unordered\_set}
tranne che possono contenere più istanze dello stesso elemento.
Per esempio nel seguente codice tutte e tre le istanze del
numero 5 vengono aggiunte al multiset:

\begin{lstlisting}
multiset<int> s;
s.insert(5);
s.insert(5);
s.insert(5);
cout << s.count(5) << "\n"; // 3
\end{lstlisting}
La funzione \texttt{erase} rimuove tutte
le istanze di un elemento da un multiset:
\begin{lstlisting}
s.erase(5);
cout << s.count(5) << "\n"; // 0
\end{lstlisting}
Quando invece deve essere rimossa una sola istanza
di un elemento da un multiset,
è possibile procedere in questo modo:
\begin{lstlisting}
s.erase(s.find(5));
cout << s.count(5) << "\n"; // 2
\end{lstlisting}

\section{Mappe}

\index{map}

Una \key{map} è un array generalizzato
che consiste di un insieme di coppie chiave-valore.
Mentre in un array normale le chiavi sono sempre
degli interi consecutivi $0,1,\ldots,n-1$,
dove $n$ è la dimensione del vettore,
le chiavi in una mappa possono essere di qualsiasi tipo
e non devono avere valori consecutivi.

La libreria standard del C++ contiene
due implementazioni di mappe, che corrispondono
alle rispettive implementazioni dei set: la struttura
\texttt{map} è basata su un albero binario bilanciato
e accedere a un elemento ha costo $O(\log n)$,
mentre la struttura
\texttt{unordered\_map} utilizza l'hashing
e l'accesso a ogni elemento costa in media $O(1)$.

Il seguente codice crea una mappa
dove le chiavi sono stringhe e i valori sono interi:

\begin{lstlisting}
map<string,int> m;
m["monkey"] = 4;
m["banana"] = 3;
m["harpsichord"] = 9;
cout << m["banana"] << "\n"; // 3
\end{lstlisting}

Se viene richiesta una chiave nella mappa
e questa non è contenuta,
viene aggiunta automaticamente con un valore 
di default che dipende dal tipo di dato.
Per esempio, nel seguente codice,
viene aggiunta alla mappa la chiave
''aybabtu'' con valore 0.

\begin{lstlisting}
map<string,int> m;
cout << m["aybabtu"] << "\n"; // 0
\end{lstlisting}

La funzione \texttt{count} controlla
se esiste una chiave nella mappa:

\begin{lstlisting}
if (m.count("aybabtu")) {
    // la chiave esiste
}
\end{lstlisting}
Il codice seguente stampa tutte le chiavi con i rispettivi valori associati:
\begin{lstlisting}
for (auto x : m) {
    cout << x.first << " " << x.second << "\n";
}
\end{lstlisting}

\section{Iteratori e range}

\index{iterator}

Molte funzioni della libreria standard del C++
usano gli iteratori.
Un \key{iteratore} è una variabile
che punta a un elemento in una struttura dati.

Spesso vengono utilizzati gli iteratori \texttt{begin}
e \texttt{end} che definiscono un range (intervallo)
che contiene tutti gli elementi di una struttura dati.
L'iteratore \texttt{begin} punta al primo elemento 
della struttura dati,
mentre l'iteratore \texttt{end} punta alla posizione \emph{dopo}
l'ultimo elemento.
Questa è la situazione che si presenta:

\begin{center}
\begin{tabular}{llllllllll}
\{ & 3, & 4, & 6, & 8, & 12, & 13, & 14, & 17 & \} \\
& $\uparrow$ & & & & & & & & $\uparrow$ \\
& \multicolumn{3}{l}{\texttt{s.begin()}} & & & & & & \texttt{s.end()} \\
\end{tabular}
\end{center}

Come si può vedere gli iteratori sono asimmetrici:
\texttt{s.begin()} punta a un elemento contenuto nella struttura dati,
mentre \texttt{s.end()} punta al di fuori della struttura dati.
Quindi l'intervallo definito dagli operatori è \emph{aperto a destra}.

\subsubsection{Lavorare con gli intervalli}

Gli iteratori sono spesso usati nelle funzioni della libreria
standard che necessitano di un intervallo all'interno di una struttura dati.
Una delle cose che si vogliono fare normalmente è elaborare
tutti gli elementi all'interno di una struttura dati,
così alla funzione verranno passati gli iteratori
\texttt{begin} e \texttt{end}.

Per esempio, il seguente codice ordina un vettore
usando la funzione \texttt{sort},
poi inverte l'ordine degli elementi usando la funzione
\texttt{reverse} e infine mescola gli elementi usando la funzione
\texttt{random\_shuffle}.

\index{sort@\texttt{sort}}
\index{reverse@\texttt{reverse}}
\index{random\_shuffle@\texttt{random\_shuffle}}

\begin{lstlisting}
sort(v.begin(), v.end());
reverse(v.begin(), v.end());
random_shuffle(v.begin(), v.end());
\end{lstlisting}

Queste funzioni possono anche essere utilizzate con normali array.
In questo caso verranno usati dei puntatori al posto degli iteratori:

\newpage
\begin{lstlisting}
//sia a ad esempio un vettore di interi
//int a[n];
sort(a, a+n);
reverse(a, a+n);
random_shuffle(a, a+n);
\end{lstlisting}

\subsubsection{Iteratori su insiemi}

Gli iteratori sono spesso usati per accedere 
agli elementi di un set.
Il seguente codice crea un iteratore
\texttt{it} che punta al più piccolo elemento di un set:
\begin{lstlisting}
set<int>::iterator it = s.begin();
\end{lstlisting}
Un modo più compatto di scrivere la stessa cosa è il seguente:
\begin{lstlisting}
auto it = s.begin();
\end{lstlisting}
L'elemento a cui punta un iteratore può essere acceduto
tramite l'operatore \texttt{*}.
Per esempio il seguente codice
stampa il più piccolo elemento nel set:

\begin{lstlisting}
auto it = s.begin();
cout << *it << "\n";
\end{lstlisting}

Gli iteratori possono essere mossi usando gli operatori
\texttt{++} (in avanti) e \texttt{--} (all'indietro),
cioè l'iteratore si sposterà all'elemento successivo (++)
o a quello precedente (- -).

Il seguente codice stampa tutti gli elementi in ordine crescente:
\begin{lstlisting}
for (auto it = s.begin(); it != s.end(); it++) {
    cout << *it << "\n";
}
\end{lstlisting}
Il seguente codice stampa il più grande elemento presente nell'insieme:
\begin{lstlisting}
auto it = s.end(); it--;
cout << *it << "\n";
\end{lstlisting}

La funzione $\texttt{find}(x)$ ritorna un iteratore che punta
all'elemento il cui valore è $x$.
Se però l'insieme non contiene $x$,
l'iteratore diventerà \texttt{end}.

\begin{lstlisting}
auto it = s.find(x);
if (it == s.end()) {
    // x non presente
}
\end{lstlisting}

La funzione $\texttt{lower\_bound}(x)$ ritorna 
un iteratore al più piccolo elemento nell'insieme
il cui valore è \emph{almeno} $x$, e 
la funzione $\texttt{upper\_bound}(x)$
ritorna un iteratore al più piccolo elemento il cui
valore è \emph{maggiore} di $x$.
Entrambe le funzioni, se tale elemento non esiste,
ritornano l'iteratore \texttt{end}.
Queste funzioni non sono implementate per 
l'\texttt{unordered\_set}, poichè questo non 
mantiene l'ordine degli elementi.

\begin{samepage}
Per esempio, il seguente codice trova l'elemento più vicino a $x$:

\begin{lstlisting}
auto it = s.lower_bound(x);
if (it == s.begin()) {
    cout << *it << "\n";
} else if (it == s.end()) {
    it--;
    cout << *it << "\n";
} else {
    int a = *it; it--;
    int b = *it;
    if (x-b < a-x) cout << b << "\n";
    else cout << a << "\n";
}
\end{lstlisting}

Questo codice assume che il set non sia vuoto,
e analizza tutti i casi possibili guardando il valore 
dell'iteratore \texttt{it}.
Siccome \texttt{it} punta al più piccolo 
valore che è almeno $x$, ci sono tre possibilità:
\begin{itemize}
\item se \texttt{it} è uguale a \texttt{begin},
l'elemento corrispondente è il più vicino a $x$ 
\item se \texttt{it} è uguale a \texttt{end},
l'elemento più grande del set è il più vicino a $x$ 
\item se non si verifica nessuno dei due casi precedenti,
l'elemento più vicino a $x$ è o l'elemento puntato da
\texttt{it} oppure l'elemento precedente.
\end{itemize}
\end{samepage}

\section{Other structures}

\subsubsection{Bitset}

\index{bitset}

A \key{bitset} is an array
whose each value is either 0 or 1.
For example, the following code creates
a bitset that contains 10 elements:
\begin{lstlisting}
bitset<10> s;
s[1] = 1;
s[3] = 1;
s[4] = 1;
s[7] = 1;
cout << s[4] << "\n"; // 1
cout << s[5] << "\n"; // 0
\end{lstlisting}

The benefit of using bitsets is that
they require less memory than ordinary arrays,
because each element in a bitset only
uses one bit of memory.
For example, 
if $n$ bits are stored in an \texttt{int} array,
$32n$ bits of memory will be used,
but a corresponding bitset only requires $n$ bits of memory.
In addition, the values of a bitset
can be efficiently manipulated using
bit operators, which makes it possible to
optimize algorithms using bit sets.

The following code shows another way to create the above bitset:
\begin{lstlisting}
bitset<10> s(string("0010011010")); // from right to left
cout << s[4] << "\n"; // 1
cout << s[5] << "\n"; // 0
\end{lstlisting}

The function \texttt{count} returns the number
of ones in the bitset:

\begin{lstlisting}
bitset<10> s(string("0010011010"));
cout << s.count() << "\n"; // 4
\end{lstlisting}

The following code shows examples of using bit operations:
\begin{lstlisting}
bitset<10> a(string("0010110110"));
bitset<10> b(string("1011011000"));
cout << (a&b) << "\n"; // 0010010000
cout << (a|b) << "\n"; // 1011111110
cout << (a^b) << "\n"; // 1001101110
\end{lstlisting}

\subsubsection{Deque}

\index{deque}

A \key{deque} is a dynamic array
whose size can be efficiently
changed at both ends of the array.
Like a vector, a deque provides the functions
\texttt{push\_back} and \texttt{pop\_back}, but
it also includes the functions
\texttt{push\_front} and \texttt{pop\_front}
which are not available in a vector.

A deque can be used as follows:
\begin{lstlisting}
deque<int> d;
d.push_back(5); // [5]
d.push_back(2); // [5,2]
d.push_front(3); // [3,5,2]
d.pop_back(); // [3,5]
d.pop_front(); // [5]
\end{lstlisting}

The internal implementation of a deque
is more complex than that of a vector,
and for this reason, a deque is slower than a vector.
Still, both adding and removing
elements take $O(1)$ time on average at both ends.

\subsubsection{Stack}

\index{stack}

A \key{stack}
is a data structure that provides two
$O(1)$ time operations:
adding an element to the top,
and removing an element from the top.
It is only possible to access the top
element of a stack.

The following code shows how a stack can be used:
\begin{lstlisting}
stack<int> s;
s.push(3);
s.push(2);
s.push(5);
cout << s.top(); // 5
s.pop();
cout << s.top(); // 2
\end{lstlisting}
\subsubsection{Queue}

\index{queue}

A \key{queue} also
provides two $O(1)$ time operations:
adding an element to the end of the queue,
and removing the first element in the queue.
It is only possible to access the first
and last element of a queue.

The following code shows how a queue can be used:
\begin{lstlisting}
queue<int> q;
q.push(3);
q.push(2);
q.push(5);
cout << q.front(); // 3
q.pop();
cout << q.front(); // 2
\end{lstlisting}

\subsubsection{Priority queue}

\index{priority queue}
\index{heap}

A \key{priority queue}
maintains a set of elements.
The supported operations are insertion and,
depending on the type of the queue,
retrieval and removal of
either the minimum or maximum element.
Insertion and removal take $O(\log n)$ time,
and retrieval takes $O(1)$ time.

While an ordered set efficiently supports
all the operations of a priority queue,
the benefit of using a priority queue is
that it has smaller constant factors.
A priority queue is usually implemented using
a heap structure that is much simpler than a
balanced binary tree used in an ordered set.

\begin{samepage}
By default, the elements in a C++
priority queue are sorted in decreasing order,
and it is possible to find and remove the
largest element in the queue.
The following code illustrates this:

\begin{lstlisting}
priority_queue<int> q;
q.push(3);
q.push(5);
q.push(7);
q.push(2);
cout << q.top() << "\n"; // 7
q.pop();
cout << q.top() << "\n"; // 5
q.pop();
q.push(6);
cout << q.top() << "\n"; // 6
q.pop();
\end{lstlisting}
\end{samepage}

If we want to create a priority queue
that supports finding and removing
the smallest element,
we can do it as follows:

\begin{lstlisting}
priority_queue<int,vector<int>,greater<int>> q;
\end{lstlisting}

\subsubsection{Policy-based data structures}

The \texttt{g++} compiler also supports
some data structures that are not part
of the C++ standard library.
Such structures are called \emph{policy-based}
data structures.
To use these structures, the following lines
must be added to the code:
\begin{lstlisting}
#include <ext/pb_ds/assoc_container.hpp>
using namespace __gnu_pbds; 
\end{lstlisting}
After this, we can define a data structure \texttt{indexed\_set} that
is like \texttt{set} but can be indexed like an array.
The definition for \texttt{int} values is as follows:
\begin{lstlisting}
typedef tree<int,null_type,less<int>,rb_tree_tag,
             tree_order_statistics_node_update> indexed_set; 
\end{lstlisting}
Now we can create a set as follows:
\begin{lstlisting}
indexed_set s;
s.insert(2);
s.insert(3);
s.insert(7);
s.insert(9);
\end{lstlisting}
The speciality of this set is that we have access to
the indices that the elements would have in a sorted array.
The function $\texttt{find\_by\_order}$ returns
an iterator to the element at a given position:
\begin{lstlisting}
auto x = s.find_by_order(2);
cout << *x << "\n"; // 7
\end{lstlisting}
And the function $\texttt{order\_of\_key}$
returns the position of a given element:
\begin{lstlisting}
cout << s.order_of_key(7) << "\n"; // 2
\end{lstlisting}
If the element does not appear in the set,
we get the position that the element would have
in the set:
\begin{lstlisting}
cout << s.order_of_key(6) << "\n"; // 2
cout << s.order_of_key(8) << "\n"; // 3
\end{lstlisting}
Both the functions work in logarithmic time.

\section{Comparison to sorting}

It is often possible to solve a problem
using either data structures or sorting.
Sometimes there are remarkable differences
in the actual efficiency of these approaches,
which may be hidden in their time complexities.

Let us consider a problem where
we are given two lists $A$ and $B$
that both contain $n$ elements.
Our task is to calculate the number of elements
that belong to both of the lists.
For example, for the lists
\[A = [5,2,8,9,4] \hspace{10px} \textrm{and} \hspace{10px} B = [3,2,9,5],\]
the answer is 3 because the numbers 2, 5
and 9 belong to both of the lists.

A straightforward solution to the problem is
to go through all pairs of elements in $O(n^2)$ time,
but next we will focus on
more efficient algorithms.

\subsubsection{Algorithm 1}

We construct a set of the elements that appear in $A$,
and after this, we iterate through the elements
of $B$ and check for each elements if it
also belongs to $A$.
This is efficient because the elements of $A$
are in a set.
Using the \texttt{set} structure,
the time complexity of the algorithm is $O(n \log n)$.

\subsubsection{Algorithm 2}

It is not necessary to maintain an ordered set,
so instead of the \texttt{set} structure
we can also use the \texttt{unordered\_set} structure.
This is an easy way to make the algorithm
more efficient, because we only have to change
the underlying data structure.
The time complexity of the new algorithm is $O(n)$.

\subsubsection{Algorithm 3}

Instead of data structures, we can use sorting.
First, we sort both lists $A$ and $B$.
After this, we iterate through both the lists
at the same time and find the common elements.
The time complexity of sorting is $O(n \log n)$,
and the rest of the algorithm works in $O(n)$ time,
so the total time complexity is $O(n \log n)$.

\subsubsection{Efficiency comparison}

The following table shows how efficient
the above algorithms are when $n$ varies and
the elements of the lists are random
integers between $1 \ldots 10^9$:

\begin{center}
\begin{tabular}{rrrr}
$n$ & Algorithm 1 & Algorithm 2 & Algorithm 3 \\
\hline
$10^6$ & $1.5$ s & $0.3$ s & $0.2$ s \\
$2 \cdot 10^6$ & $3.7$ s & $0.8$ s & $0.3$ s \\
$3 \cdot 10^6$ & $5.7$ s & $1.3$ s & $0.5$ s \\
$4 \cdot 10^6$ & $7.7$ s & $1.7$ s & $0.7$ s \\
$5 \cdot 10^6$ & $10.0$ s & $2.3$ s & $0.9$ s \\
\end{tabular}
\end{center}

Algorithms 1 and 2 are equal except that
they use different set structures.
In this problem, this choice has an important effect on
the running time, because Algorithm 2
is 4–5 times faster than Algorithm 1.

However, the most efficient algorithm is Algorithm 3
which uses sorting.
It only uses half the time compared to Algorithm 2.
Interestingly, the time complexity of both
Algorithm 1 and Algorithm 3 is $O(n \log n)$,
but despite this, Algorithm 3 is ten times faster.
This can be explained by the fact that
sorting is a simple procedure and it is done
only once at the beginning of Algorithm 3,
and the rest of the algorithm works in linear time.
On the other hand,
Algorithm 1 maintains a complex balanced binary tree
during the whole algorithm.
